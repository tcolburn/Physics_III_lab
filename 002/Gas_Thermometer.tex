\documentclass{revtex4-1}
\usepackage{graphicx}% Include figure files
\usepackage{dcolumn}% Align table columns on decimal point
\usepackage{bm}% bold math
\usepackage{geometry}[1in]
\usepackage{amsmath}
\usepackage{relsize}

\author{Kate Hoverson, Mack Baker, Nathaniel Chrisman, Taylor Colburn}
\title{Linearity of ideal gas state variables}
\date{\today}
\begin{document}
  \maketitle
  \begin{abstract}
    A volume of a fixed mass of air and a volume of air at a fixed atmospheric pressure were measured as a function of temperature. The relationship between pressure, temperature and volume was studied using the ideal gas law formula. The data collected was analyzed using the ideal gas law to determine absolute zero temperature.
  \end{abstract}
  \section{Introduction}

  The determination of absolute value by obtaining temperature readings of a volume of gas using two types of thermometers is the goal of this experiment.  By holding a certain parameters of the ideal gas law constant, the data will be compared to the model to confirm the relationship of pressure, volume, and temperature.  The first thermometer holds atmospheric pressure constant and the change in volume will indicate the temperature. The second apparatus allows pressure to change while the volume is held at a constant. The measured temperature of the mass of air at a fixed volume should provide sufficient data for the linearity of the temperature as a function of pressure to be determined.

  The ideal gas law is:

  $$PV = nRT$$

  Where,

  $$P = \text{pressure}$$
  $$V = \text{volume}$$
  $$n = \text{number of moles}$$
  $$R = \text{gas constant}$$
  $$T = \text{temperature}$$


Should all these variables be measured with accuracy, the model should provide enough data for a linear regression model to determine absolute zero temperature within a minimal margin of experimental error.
